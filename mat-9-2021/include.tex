\geometry{
  a4paper,
  includeheadfoot,
  lmargin=\margemesq,
  rmargin=\margemdir,
  tmargin=\margemtopo,
  bmargin=\margembaix,
  headheight=\altcabec,
}

%%%%%%%%%% Cabeçalho e rodapé %%%%%%%%%%%%%%%%

\fancyhf{}
\fancypagestyle{primeirapagina}{
  \setlength{\headheight}{0mm}
  \setlength{\headsep}{0mm}
  \fancyhead[LE,RO]{\includegraphics[width=2.3cm]{logo}}
  \fancyfoot[LE,RO]{\thepage}
}

\fancypagestyle{normal}{
  \fancyhead[LE,RO]{\includegraphics[width=2.3cm]{logo}}
  \fancyfoot[LE,RO]{\thepage}
}

\pagestyle{normal}

%%%%%%%%%%%%%%%%%%%%%%%%%%%%%%%%%%%%%%%%%%%%%%

% CABEÇALHO ARCO (provavelmente você não quer mexer nisso)
% titulo
\newlength{\largtit} \setlength{\largtit}{105mm}
\newlength{\alttit} \setlength{\alttit}{8mm}
% infos
\newlength{\larginfo} \setlength{\larginfo}{175mm}
\newlength{\divisaoinfo} \setlength{\divisaoinfo}{120mm}
\newlength{\altinfo} \setlength{\altinfo}{5mm}

% Desconta espaço horizontal (posição do cabeçalho arco é fixa)
\newlength{\subtraih} \setlength{\subtraih}{-\margemesq + 0.75cm}

\renewcommand{\headrulewidth}{0pt} % não usar a linha padrão do cabeç.
\renewcommand{\footrulewidth}{0pt}

\newcommand{\cabecalho}[4] {
  \thispagestyle{primeirapagina}
  \hspace*{\subtraih}
  % \vspace*{\subtraiv}
  \begin{overpic}[abs, unit=1mm, width=16.347cm + 2cm]{cabecalho}
    % Título
    \put (3.3,22) {
      \begin{minipage}[c][\alttit][c]{\largtit}
        #1
      \end{minipage}
    }
    % Subtítulo
    \put (3.3,12.9) {
      \begin{minipage}[c][\alttit][c]{\largtit}
        #2
      \end{minipage}
    }
    % Informações
    \put (3.3,3.7) {
      \begin{minipage}[c][\altinfo][c]{\divisaoinfo}
        #3
      \end{minipage}
      \begin{minipage}[c][\altinfo][c]{\larginfo - \divisaoinfo}
          #4
      \end{minipage}
    }
  \end{overpic}
  \hspace*{0mm}
  \bigskip
}

